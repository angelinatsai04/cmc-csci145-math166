\documentclass[10pt]{exam}

\usepackage[margin=1in]{geometry}
\usepackage{amsmath}
\usepackage{amssymb}
\usepackage{amsthm}
\usepackage{mathtools}
\usepackage{bm}
\usepackage{stmaryrd}
\usepackage{booktabs}
\usepackage{tikz}

\usepackage{color}
\usepackage{colortbl}
\definecolor{deepblue}{rgb}{0,0,0.5}
\definecolor{deepred}{rgb}{0.6,0,0}
\definecolor{deepgreen}{rgb}{0,0.5,0}
\definecolor{gray}{rgb}{0.7,0.7,0.7}

\usepackage{hyperref}
\hypersetup{
  colorlinks   = true, %Colours links instead of ugly boxes
  urlcolor     = black, %Colour for external hyperlinks
  linkcolor    = blue, %Colour of internal links
  citecolor    = blue  %Colour of citations
}

\usepackage{listings}

%%%%%%%%%%%%%%%%%%%%%%%%%%%%%%%%%%%%%%%%%%%%%%%%%%%%%%%%%%%%%%%%%%%%%%%%%%%%%%%%

\theoremstyle{definition}
\newtheorem{problem}{Problem}
\newtheorem{note}{Note}
\newtheorem{example}{Example}
\newtheorem{defn}{Definition}
\newtheorem{fact}{Fact}
\newtheorem{refr}{References}
\newtheorem{theorem}{Theorem}
\newcommand{\E}{\mathbb E}
\newcommand{\R}{\mathbb R}
\DeclareMathOperator{\nnz}{nnz}
\DeclareMathOperator{\sign}{sign}
\DeclareMathOperator{\determinant}{det}
\DeclareMathOperator{\Var}{Var}
\DeclareMathOperator{\rank}{rank}
\DeclareMathOperator{\prob}{\mathbb P}
\DeclareMathOperator*{\argmin}{arg\,min}
\DeclareMathOperator*{\argmax}{arg\,max}

\newcommand{\Ein}{E_{\text{in}}}
\newcommand{\Eout}{E_{\text{out}}}
\newcommand{\Etest}{E_{\text{test}}}
\newcommand{\I}{\mathbf I}
\newcommand{\Q}{\mathbf Q}
\newcommand{\p}{\mathbf P}
\newcommand{\pb}{\bar {\p}}
\newcommand{\pbb}{\bar {\pb}}
\newcommand{\pr}{\bm \pi}

\newcommand{\trans}[1]{{#1}^{T}}
\newcommand{\loss}{\ell}
\newcommand{\w}{\mathbf w}
\newcommand{\wstar}{{\w}^{*}}
\newcommand{\x}{\mathbf x}
\newcommand{\y}{\mathbf y}
\newcommand{\lone}[1]{{\lVert {#1} \rVert}_1}
\newcommand{\ltwo}[1]{{\lVert {#1} \rVert}_2}
\newcommand{\lp}[1]{{\lVert {#1} \rVert}_p}
\newcommand{\linf}[1]{{\lVert {#1} \rVert}_\infty}
\newcommand{\lF}[1]{{\lVert {#1} \rVert}_F}

\newcommand{\mH}{m_{\mathcal H}}
\newcommand{\dvc}{{d_{\text{VC}}}}
\newcommand{\HH}[1]{\mathcal H_{\text{#1}}}
\newcommand{\Hbinary}{\HH_{\text{binary}}}
\newcommand{\Haxis}{\HH_{\text{axis}}}
\newcommand{\Hperceptron}{\HH_{\text{perceptron}}}


\newcommand{\ignore}[1]{}

%%%%%%%%%%%%%%%%%%%%%%%%%%%%%%%%%%%%%%%%%%%%%%%%%%%%%%%%%%%%%%%%%%%%%%%%%%%%%%%%

\begin{document}


\begin{center}
{
\Huge
Chapter 3: Linear Models
}

    \vspace{0.05in}
{
\Large
Sections 3.1-3.3
    (Temporarily skipping Section 2.3.)
}
\end{center}

%\begin{defn}
    %A loss function $\ell(\hat y, y)$ measures the cost of predicting $\hat y$ when the actual answer is $y$.
%\end{defn}

\section*{Motivation}

%\begin{defn}
Recall that a model is defined both by its hypothesis class $\HH{}$ and the training algorithm used to select a hypothesis from $\HH{}$ on a particular dataset.
%The ``best possible'' training algorithm will compute the minimization
A reasonable idea for a training algorithm is to select a hypothesis that minimizes the in-sample error:
\begin{equation}
    \label{eq:erm}
    g = \argmin_{h\in\HH{}} \Ein(h).
\end{equation}
%The hypothesis described by this process above is often called the \emph{emperical risk minimizer} (ERM) because $\Ein$ is often called the \emph{emperical risk}.
%
    For finite hypothesis classes, we saw that the \emph{try everything algorithm} (TEA) can exactly compute the minimization in \eqref{eq:erm} in finite time.
    Unfortunately, the TEA algorithm cannot be used on infinite hypothesis classes.

    In this set of notes, we will study how to compute the minimization above for infinite classes.
    %We follow a three step procedure:
    %\begin{enumerate}
        %\item We'll introduce new loss functions that are easy to work with.
        %\item We'll describe how they statistically compare to the 0-1 loss we've already seen.
        %\item We'll introduce algorithms for 
    %\end{enumerate}
%
    %The high level idea is that we will introduce new loss functions that make it easy to solve the optimization in \eqref{eq:erm} above.
    %Then we will describe how they relate statistically to the 0-1 loss we've already seen.

\section*{Notation}

\begin{note}
The textbook has slightly sloppy notation in this chapter.
The purpose of this sloppiness is to try to make the material more approachable by omitting certain dependencies.
I have found that this confuses students, however, and so am providing a less sloppy version of the notation below.
Either versions of these notations are acceptable on exams.
\end{note}

\begin{note}
Sections 2.2.4 and 2.3 from the previous chapter is also related to the notation below.
It should be read before Section 3.
\end{note}

\begin{defn}
    A \emph{loss function} $\ell(\hat y, y)$ is a function that measures ``how bad'' the output of a hypothesis is.
    The \emph{in-sample error} with respect to $\ell$ is defined to be
    \begin{equation}
        \label{eq:Ein}
        \Ein^{\ell}(h) = \frac{1}{N} \sum_{i=1}^N \ell\big(h(\x_i), y_i\big)
    \end{equation}
    and the \emph{out-of-sample error} with respect to $\ell$ is
    \begin{equation}
        \label{eq:Eout}
        \Eout^{\ell}(h) = \prob \ell\big(h(\x), y\big)
        .
    \end{equation}
\end{defn}

\begin{example}
    The $\emph{0-1 loss}$ is defined to be
    \begin{equation}
        \label{eq:01}
        \ell_{0-1}(\hat y, y) = \llbracket \hat y \ne y \rrbracket
        .
    \end{equation}
    Substituting \eqref{eq:01} into \eqref{eq:Eout} and \eqref{eq:Ein} above yield the familiar definitions for $\Ein$ and $\Eout$ from Chapter 1.
\end{example}

\newpage
\section*{The Squared Loss (Section 3.2)}

\begin{example}
    In a regression problem, the goal is to predict a real number instead of a binary class label.
    That is, the output space is $\mathcal Y=\R$ instead of $\mathcal Y = \{+1, -1\}$.
    A popular model for regression is \emph{ordinary least squares} (OLS), which uses the linear hypothesis class
    \begin{equation}
        \HH{} = \bigg\{ \x \mapsto \trans\w \x : \w \in \R^d \bigg\}.
    \end{equation}
    It is common to use the \emph{squared loss} when evaluating regression problems.
    It is defined to be
    \begin{equation}
        \label{eq:l2loss}
        \ell(\hat y, y) = (\hat y - y)^2
        .
    \end{equation}
    Substituting \eqref{eq:l2loss} into \eqref{eq:Ein} gives the equation for the in-sample error
    \begin{equation}
        \Ein(h) 
        = \frac{1}{N}\sum_{i=1}^N (\trans\x_i \w - y_i)^2
        = \ltwo{X\w - Y}^2,
    \end{equation}
    where $X$ is a $N \times d$ matrix with $i$th row equal to $\x_i$ and $Y$ is the $N$ dimensional vector with $i$th position equal to $y_i$.
\end{example}

%\begin{problem}
    %Under what conditions is this minimizer defined?
%\end{problem}

%\newpage
\begin{problem}
    The OLS model is convenient because the hypothesis $g$ which minimizes $\Ein$ can be computed in closed form.
    Show this computation.

    HINT: WolframAlpha can't take matrix/vector derivatives.
    But \url{https://www.matrixcalculus.org} can take these higher-order derivatives.
    %\begin{equation}
        %\hat \w = \argmin_{\w \in \R^d} \ltwo{X\w - Y}^2
    %\end{equation}
\end{problem}

\newpage
\begin{problem}
    What is the runtime of computing the ERM for OLS?
\end{problem}

%\newpage
\vspace{4in}
%\section*{Section 2.3: Approximation-Generalization Tradeoff}
\begin{problem}
    The VC dimension is undefined for the OLS problem.
    Why?
\end{problem}

%\vspace{4in}
%\begin{problem}
    %Define the \emph{bias} end \emph{variance}
    %\begin{align*}
        %\text{bias}(\x) & = (\bar g - f(x))^2 \\
    %\end{align*}
%\end{problem}

%\vspace{3in}
%\begin{note}
    %The textbook bounds the generalization error of the squared loss in this section and Chapter 2.3.
    %These bounds are simpler than the VC dimension bounds to understand and prove.
    %We will not cover them in this course, however,
    %because they require knowledge of probability that is (unfortunately IMNSHO) not a pre-requisite for this course.
    %These bounds are often expected knowledge in technical machine learning interviews.
%\end{note}

\newpage
\begin{fact}
The squared loss is essentially unique in having a closed form minimizer.
Most other loss functions (and all other loss functions used in practuce) do not have closed form minimizers and require iterative procedures to solve.
\end{fact}

\section*{The 0-1 Loss (Section 3.1)}

%\begin{example}
What you need to know:
\begin{enumerate}
    \item
    The 0-1 loss used by the perceptron has no closed form solution.
    \item
    In chapter 1, we introduced the PLA algorithm for computing a hypothesis $g$ that minimizes $\Ein$ only when the data is linearly separable.
    When the data is not linearly separable, the PLA will never terminate.
    (You should review the PLA algorithm and ensure you understand why.)

    \item
    In Section 3.1, the book introduces the \emph{pocket algorithm} for computing a hypothesis $g$ when the data is not linearly separable.
    Unfortunately, this algorithm is not guaranteed to converge to a minimizer of $\Ein$.
    You should be able to explain why.

    \item
    In general, minimizing the 0-1 loss is \emph{NP-hard}.

\begin{fact}
    The best algorithms that we have for solving NP-hard problems currently take exponential time.
    In the case of minimizing the 0-1 loss, that means that the best known algorithms take time $\Omega((1+\epsilon)^d)$ where $\epsilon$ is a small constant greater than 0. 
    Most people believe that minimizing the 0-1 loss cannot be done in sub-exponential time,
    but we do not currently have a proof that this is the case.
    Therefore, finding the optimal runtime for directly minimizing the 0-1 loss is an \textbf{open problem}.
    There are no meaningful known lower bounds on the optimal runtime.

    If you haven't studied NP-hard problems before, I strongly recommend you watch the video \url{https://www.youtube.com/watch?v=YX40hbAHx3s}.
\end{fact}

\end{enumerate}
%\end{example}

\newpage
\section*{The Logistic Loss (Section 3.3)}

Since the 0-1 loss cannot be efficiently optimized, our goal is to introduce new loss functions that are ``similar'' to the 0-1 loss but can be efficiently optimized.

Recall that the 0-1 loss was previously defined to be
\begin{equation}
    \ell_{0-1}(\hat y, y) = \llbracket \hat y \ne y \rrbracket
    %= -\sign(\hat y y)
    .
\end{equation}
where $\hat y$ and $y$ are restricted to be in the set $\{-1, 1\}$.
We can generalize this definition to allow the $\hat y$ variable to be any real number as follows
\begin{equation}
    \ell_{0-1}(\hat y, y)
    =
    \begin{cases}
        0 & \text{if}~\hat y y > 0 \\
        1 & \text{otherwise} \\
    \end{cases}
\end{equation}
and these two definitions agree whenever they are both defined.

\begin{tikzpicture}
\draw[<->, thick] (-8,0) -- (8,0) node[right] {};
\draw[->, thick] (0,-1) -- (0,7) node[above] {};

\foreach \x in {-3,0,3, 6, -6} {
    \draw[dashed] (\x,-1) -- (\x,7);
}
\foreach \y in {0,3,6} {
    \draw[dashed] (-8,\y) -- (8,\y);
}
\end{tikzpicture}

\vspace{0.1in}
\begin{tikzpicture}
\draw[<->, thick] (-8,0) -- (8,0) node[right] {};
\draw[->, thick] (0,-1) -- (0,7) node[above] {};

\foreach \x in {-3,0,3, 6, -6} {
    \draw[dashed] (\x,-1) -- (\x,7);
}
\foreach \y in {0,3,6} {
    \draw[dashed] (-8,\y) -- (8,\y);
}
\end{tikzpicture}


%\vspace{1in}
\newpage
\begin{defn}
The quantity $z = \hat y y$ appears commonly in classification losses, and so classification losses are commonly written as functions of a single variable.
The 0-1 loss can be equivalently defined in this way as
\begin{equation}
    Q_{0-1}(z)
    =
    \begin{cases}
        0 & \text{if}~z > 0 \\
        1 & \text{otherwise} \\
    \end{cases}
\end{equation}
\end{defn}

\begin{defn}
    A \emph{surrogate loss function} $\ell$ is any loss function that upper bounds the 0-1 loss.
    That is,
    %\begin{equation}
        %\ell(\hat y, y) \ge \ell_{0-1}(\hat y, y)
    %\end{equation}
    %for all values of $\hat y$ and $y$,
    %or equivalently
    %\begin{equation}
        %\ell(\hat y, y) \ge \ell_{0-1}(\hat y, y)
        $Q(z) \ge Q_{0-1}(z)$
    %\end{equation}
    for all values of $z$.
\end{defn}

\begin{defn}[informal]
    A loss function is \emph{convex} if the line segment between any two points on the graph of the function lies above the graph between the two points.
\end{defn}

\begin{defn}[informal]
    The closer a loss function is to the 0-1 loss, the more \emph{robust} that loss function is.
    The more robust a loss function is, the closer its minimizer will be to the hard-to-compute minimizer of the 0-1 loss and the less influence ``outliers'' will have on the loss.
\end{defn}

%\vspace{1in}
\begin{problem}
    Complete the table below, and draw each loss function.

    \vspace{0.15in}
    \renewcommand{\arraystretch}{2.5}
    \begin{tabular}{llp{1in}p{1in}p{1in}}
        \toprule
        loss name & $Q(z)$ & surrogate? & convex? & robust?\\
        \midrule
        exponential loss & $\exp(-z)$ \\
        squared loss & $(1-z)^2$ \\
        logistic loss & $\log(1 + \exp(-z))$ \\
        hinge loss & $\max\{0, 1-z\}$ \\
        trimmed hinge loss & $\min\big\{2, \max\{0, 1-z\}\big\}$ \\
        smooth 0-1 loss & $\tfrac{1}{2} + \arctan(-\alpha \cdot z)/\pi$ \\
        \bottomrule
    \end{tabular}
    %\begin{tabular}{ll}
        %\toprule
        %name & $\ell(\hat y, y)$ \\
        %\midrule
        %squared loss & $(1-\hat y y)^2$ \\
        %exponential loss & $\exp(-\hat y y)$ \\
        %logistic loss & $\log(1 + \exp(-\hat y y))$ \\
        %hinge loss & $\max\{0, 1-\hat y y\}$ \\
        %trimmed hinge loss & $\min\big\{2, \max\{0, 1-\hat y y\}\big\}$ \\
        %smooth 0-1 loss & $\tfrac{1}{2} + \arctan(-\alpha \cdot \hat y y)/\pi$ \\
        %\bottomrule
    %\end{tabular}
\end{problem}
\begin{tikzpicture}
\draw[<->, thick] (-8,0) -- (8,0) node[right] {};
\draw[->, thick] (0,-1) -- (0,7) node[above] {};

\foreach \x in {-3,0,3, 6, -6} {
    \draw[dashed] (\x,-1) -- (\x,7);
}
\foreach \y in {0,3,6} {
    \draw[dashed] (-8,\y) -- (8,\y);
}

\end{tikzpicture}

\newpage
%\begin{problem}
    %Draw the loss functions
%\end{problem}

\begin{problem}
    Let $\ell$ be a surrogate loss function for $\ell_{0-1}$.
    Derive a VC generalization bound for the generalization error $\Eout^{\ell_{0-1}} - \Ein^{\ell}$.
\end{problem}

%\begin{fact}
    %Every surrogate loss function $\ell$ has the property that
    %\begin{equation}
        %\label{eq:upper}
        %\Ein^{\ell_{0-1}} \le \Ein^{\ell}
    %\end{equation}
    %and so we can use VC theory to get a generalization bound on the classification error for any model trained with a surrogate loss.
    %In particular, substituting \eqref{eq:upper} into the VC theorem gives
    %\begin{equation}
        %\label{eq:loss_ftsl}
        %\Eout^{\ell_{0-1}} \le \Ein^{\ell} + O\bigg(\sqrt{\frac{\dvc\log N}{N}}\bigg)
    %\end{equation}
    %with high probability.
    %In particular, \eqref{eq:loss_ftsl} holds for the logistic regression model (using the logistic loss) and the SVM model (using the hinge loss).
%\end{fact}
%
%%\begin{note}
    %%The textbook also discusses a probabilistic interpretation of the logistic loss.
    %%Again, due to the course prereqs, we will not cover this interpretation,
    %%but this is sometimes material that is asked about in technical interviews.
%%\end{note}
%
%\section*{Optimizing the Surrogate Loss}
%
%Section 3.3.2 of the textbook has an ``intuitive'' overview of the gradient descent and stochastic gradient descent algorithms.
%This is not detailed enough, and so we will supplement with L\'eon Bottou's paper ``Large-Scale Machine Learning with Stochastic Gradient Descent.''
%
%Paraskavedekatriaphobia

\end{document}



