\documentclass[10pt]{exam}

\usepackage[margin=1in]{geometry}
\usepackage{amsmath}
\usepackage{amssymb}
\usepackage{amsthm}
\usepackage{mathtools}
\usepackage{bm}
\usepackage{stmaryrd}
\usepackage{booktabs}

\usepackage{color}
\usepackage{colortbl}
\definecolor{deepblue}{rgb}{0,0,0.5}
\definecolor{deepred}{rgb}{0.6,0,0}
\definecolor{deepgreen}{rgb}{0,0.5,0}
\definecolor{gray}{rgb}{0.7,0.7,0.7}

\usepackage{hyperref}
\hypersetup{
  colorlinks   = true, %Colours links instead of ugly boxes
  urlcolor     = black, %Colour for external hyperlinks
  linkcolor    = blue, %Colour of internal links
  citecolor    = blue  %Colour of citations
}

\usepackage{listings}
\lstset{
    basicstyle={\ttfamily}
}

%%%%%%%%%%%%%%%%%%%%%%%%%%%%%%%%%%%%%%%%%%%%%%%%%%%%%%%%%%%%%%%%%%%%%%%%%%%%%%%%

\theoremstyle{definition}
\newtheorem{problem}{Problem}
\newtheorem{example}{Example}
\newtheorem{lemma}{Lemma}
\newtheorem{corollary}{Corollary}
\newtheorem{note}{Note}
\newtheorem{defn}{Definition}
\newtheorem{fact}{Fact}
\newtheorem{refr}{References}
\newtheorem{theorem}{Theorem}
\newcommand{\E}{\mathbb E}
\newcommand{\R}{\mathbb R}
\DeclareMathOperator{\nnz}{nnz}
\DeclareMathOperator{\sign}{sign}
\DeclareMathOperator{\determinant}{det}
\DeclareMathOperator{\Var}{Var}
\DeclareMathOperator{\rank}{rank}
\DeclareMathOperator{\prob}{\mathbb P}
\DeclareMathOperator*{\argmin}{arg\,min}
\DeclareMathOperator*{\argmax}{arg\,max}

\newcommand{\Ein}{E_{\text{in}}}
\newcommand{\Eout}{E_{\text{out}}}
\newcommand{\Etest}{E_{\text{test}}}
\newcommand{\I}{\mathbf I}
\newcommand{\Q}{\mathbf Q}
\newcommand{\p}{\mathbf P}
\newcommand{\pb}{\bar {\p}}
\newcommand{\pbb}{\bar {\pb}}
\newcommand{\pr}{\bm \pi}

\newcommand{\trans}[1]{{#1}^{T}}
\newcommand{\loss}{\ell}
\newcommand{\w}{\mathbf w}
\newcommand{\wstar}{{\w}^{*}}
\newcommand{\x}{\mathbf x}
\newcommand{\y}{\mathbf y}
\newcommand{\lone}[1]{{\lVert {#1} \rVert}_1}
\newcommand{\ltwo}[1]{{\lVert {#1} \rVert}_2}
\newcommand{\lp}[1]{{\lVert {#1} \rVert}_p}
\newcommand{\linf}[1]{{\lVert {#1} \rVert}_\infty}
\newcommand{\lF}[1]{{\lVert {#1} \rVert}_F}

\newcommand{\mH}{m_{\mathcal H}}
\newcommand{\dvc}{{d_{\text{VC}}}}
\newcommand{\HH}[1]{\mathcal H_{\text{#1}}}
\newcommand{\Hbinary}{\HH_{\text{binary}}}
\newcommand{\Haxis}{\HH_{\text{axis}}}
\newcommand{\Hperceptron}{\HH_{\text{perceptron}}}


\newcommand{\ignore}[1]{}

%%%%%%%%%%%%%%%%%%%%%%%%%%%%%%%%%%%%%%%%%%%%%%%%%%%%%%%%%%%%%%%%%%%%%%%%%%%%%%%%

\begin{document}


\begin{center}
{
\Huge
Model Zoo 3
}
\end{center}

\noindent
The textbook has detailed explanations of all of the models we've covered in the model zoo in the online supplements located at

\vspace{0.15in}
\url{https://amlbook.com/eChapters.html}

\vspace{0.15in}
\noindent
The online supplements are all password protected with password \lstinline{Paraskavedekatriaphobia}.
For the most part, these online supplements are technical and provide relatively little insight into practical applications.
The section on nearest neighbor algorithms (e-Chapter 6), however, is relatively readable.

\section*{Nearest Neighbor Methods}

\begin{problem}
Describe the $k$-nearest neighbor classifier.
\end{problem}

%\begin{theorem}
    %We have a known lower bound on the fastest $\Omega(2^d \log N)$.
%\end{theorem}

\newpage
\begin{fact}
    The in sample error of 1-nearest neighbor is always 0.
\end{fact}

%\begin{fact}
    %The VC-dimension of $k$-nearest neighbor methods is infinite.
%\end{fact}

\vspace{2in}
\begin{theorem}[informal]
    Let $h$ be the 1-NN hypothesis.
    Then for ``well behaved'' data distributions,
    we have with high probability that
    \begin{equation}
        \Eout(h) \le 2 \Eout(f) + 4 \sqrt{d}N^{-\frac{1}{d+1}}.
    \end{equation}
    If $h$ is the $k$-nearest neighbor classifier,
    then with high probability,
    \begin{equation}
        \Eout(h) \ge (1 + 1/k) \Eout(f)
        .
    \end{equation}
\end{theorem}
\begin{proof}
See Theorem 19.3 of \emph{Understanding Machine Learning: From Theory to Algorithms}.
\end{proof}

\newpage
\begin{problem}
    Recall that the ImageNet dataset has an estimated Bayes Error around 5\%.
    (This estimate follows from Karpathy's observed human error rate of 5\% and the observed labelled errors from the automatic collection process.)
    What is the best out of sample error we can expect from a 1-NN classifier?
\end{problem}

\vspace{3in}
\begin{problem}
    Recall that the MNIST digit classification dataset has $N=60000$ training image, and each image has $d=28\times28=784$ dimensions.
    There are 13 known mislabeled images.
    (You can see them at \url{https://labelerrors.com/}.)
    \begin{enumerate}
        \item Why 1-NN is a bad choice of algorithm for this dataset?
        \vspace{2in}
        \item Would it make more sense to use PCA or the polynomial feature map on this dataset before performing 1-NN?
    \end{enumerate}
\end{problem}

%\newpage
%\begin{note}
%You are responsible for understanding how the following hyperparameters affect
%
    %\noindent\lstinline{sklearn.neighbors.KNeighborsClassifier}.
    %\begin{enumerate}
        %\item \lstinline{n_neighbors}
        %\item \lstinline{weights}
        %\item \lstinline{algorithm}
        %\item \lstinline{metric}
    %\end{enumerate}
%\end{note}
%\begin{problem}
    %You have a dataset of size $N=10^5$ with $d=4$ dimensions.
%\end{problem}

\newpage
\section*{Support Vector Machines (SVMs)}

There are three standard and equivalent interpretations of SVMs:
\begin{enumerate}
\item The hypothesis class $\HH{perceptron}$ with the hinge loss $\loss(\x;\w) = \max(1-\trans\w\x, 0)$.
\item The large margin classifier.
\item A smooth generalization of the k-nearest neighbor algorithm.
\end{enumerate}

\newpage
\begin{note}
The following theorem is copied directly from Theorem 8.5 in Chapter 8 of the \emph{Learning from Data} textbook.
Most of this chapter is more technical than needed for this course,
but the explanation of the theorem is straightforward and worth reading.
\end{note}
\begin{theorem}
    Suppose the input space is the ball of radius $R$ in $\R^d$.
    That is, $\ltwo{\x} \le R$.
    Then,
    \begin{equation}
        \dvc \le \lceil R^2/\rho^2\rceil + 1.
    \end{equation}
\end{theorem}

\newpage
\begin{problem}
    Describe the \emph{dual learning problem} and the \emph{kernel trick}.
\end{problem}
%\section{Non-binary Classification}


\newpage
%\begin{problem}
    %Describe how changes to the following hyperparameters of \lstinline{sklearn.svm.SVC} affect model performance.
    %\begin{enumerate}
        %\item \lstinline{C}
        %\item \lstinline{kernel}
        %\item \lstinline{degree}
        %\item \lstinline{gamma}
        %\item \lstinline{coef0}
        %\item \lstinline{tol}
        %\item \lstinline{cache_size}
        %\item \lstinline{max_iter}
    %\end{enumerate}
%\end{problem}

\noindent
Common sample kernel functions include:

\vspace{0.15in}
{
\setlength{\tabcolsep}{35pt}
\renewcommand{\arraystretch}{3.5}
\begin{tabular}{lcc}
    \hline
    kernel name & $K(\x_1, \x_2)$ & feature dimensions $(\tilde d)$ \\
    \hline
    linear & $\trans\x_1 \x_2$ & $d$ \\
    polynomial & $(\gamma\trans\x_1\x_2 + r)^Q$ & $\Theta(d^Q)$ \\
    gaussian & $\exp(-\gamma\ltwo{\x_1-\x_2}^2)$ & $\infty$ \\
    sigmoid & $\tanh(\gamma\trans\x_1\x_2 + r)$ & $\infty$ \\
    \hline
\end{tabular}
}

\end{document}



