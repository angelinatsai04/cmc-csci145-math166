\documentclass[10pt]{exam}

\usepackage[margin=1in]{geometry}
\usepackage{amsmath}
\usepackage{amssymb}
\usepackage{amsthm}
\usepackage{mathtools}
\usepackage{bm}
\usepackage{stmaryrd}
\usepackage{booktabs}
\usepackage{array}
\newcolumntype{C}[1]{>{\centering\arraybackslash\hspace{0pt}}p{#1}}

\usepackage{color}
\usepackage{colortbl}
\definecolor{deepblue}{rgb}{0,0,0.5}
\definecolor{deepred}{rgb}{0.6,0,0}
\definecolor{deepgreen}{rgb}{0,0.5,0}
\definecolor{gray}{rgb}{0.7,0.7,0.7}

\usepackage{hyperref}
\hypersetup{
  colorlinks   = true, %Colours links instead of ugly boxes
  urlcolor     = black, %Colour for external hyperlinks
  linkcolor    = blue, %Colour of internal links
  citecolor    = blue  %Colour of citations
}

\usepackage{listings}

%%%%%%%%%%%%%%%%%%%%%%%%%%%%%%%%%%%%%%%%%%%%%%%%%%%%%%%%%%%%%%%%%%%%%%%%%%%%%%%%

\theoremstyle{definition}
\newtheorem{problem}{Problem}
\newtheorem{note}{Note}
\newtheorem{example}{Example}
\newtheorem{defn}{Definition}
\newtheorem{fact}{Fact}
\newtheorem{refr}{References}
\newtheorem{theorem}{Theorem}
\newcommand{\E}{\mathbb E}
\newcommand{\R}{\mathbb R}
\DeclareMathOperator{\nnz}{nnz}
\DeclareMathOperator{\sign}{sign}
\DeclareMathOperator{\determinant}{det}
\DeclareMathOperator{\Var}{Var}
\DeclareMathOperator{\rank}{rank}
\DeclareMathOperator{\prob}{\mathbb P}
\DeclareMathOperator*{\argmin}{arg\,min}
\DeclareMathOperator*{\argmax}{arg\,max}

\newcommand{\Ein}{E_{\text{in}}}
\newcommand{\Eout}{E_{\text{out}}}
\newcommand{\Etest}{E_{\text{test}}}
\newcommand{\Etrain}{E_{\text{train}}}
\newcommand{\Eval}{E_{\text{val}}}
\newcommand{\Eaug}{E_{\text{aug}}}
\newcommand{\Dtrain}{\mathcal D_{\text{train}}}
\newcommand{\Dval}{\mathcal D_{\text{val}}}
\newcommand{\I}{\mathbf I}
\newcommand{\Q}{\mathbf Q}
\newcommand{\p}{\mathbf P}
\newcommand{\pb}{\bar {\p}}
\newcommand{\pbb}{\bar {\pb}}
\newcommand{\pr}{\bm \pi}

\newcommand{\trans}[1]{{#1}^{T}}
\newcommand{\loss}{\ell}
\newcommand{\w}{\mathbf w}
\newcommand{\wstar}{{\w}^{*}}
\newcommand{\x}{\mathbf x}
\newcommand{\y}{\mathbf y}
\newcommand{\lone}[1]{{\lVert {#1} \rVert}_1}
\newcommand{\ltwo}[1]{{\lVert {#1} \rVert}_2}
\newcommand{\lp}[1]{{\lVert {#1} \rVert}_p}
\newcommand{\linf}[1]{{\lVert {#1} \rVert}_\infty}
\newcommand{\lF}[1]{{\lVert {#1} \rVert}_F}

\newcommand{\mH}{m_{\mathcal H}}
\newcommand{\dvc}{{d_{\text{VC}}}}
\newcommand{\HH}[1]{\mathcal H_{\text{#1}}}
\newcommand{\Hbinary}{\HH_{\text{binary}}}
\newcommand{\Haxis}{\HH_{\text{axis}}}
\newcommand{\Hperceptron}{\HH_{\text{perceptron}}}


\newcommand{\ignore}[1]{}

%%%%%%%%%%%%%%%%%%%%%%%%%%%%%%%%%%%%%%%%%%%%%%%%%%%%%%%%%%%%%%%%%%%%%%%%%%%%%%%%

\begin{document}


\begin{center}
{
\Huge
Chapter 4.3: Validation / Model Selection
}
\end{center}

%We paused our discussion about Chapter 4 (Overfitting) of the textbook in order to talk about more types of models (Model Zoo/Transfer Learning).
%We now revisit Chapter 4

Model selection is the processes of selecting the best hyperparameters for a learning problem.
It is the most important step in real-world data mining tasks.

One of the weaknesses of our textbook is that it does not discuss in detail very many models to select hyper-parameters for.
So in order to have more interesting models to work with,
we spent the last few lectures covering the ``Model Zoo'' and ``Transfer Learning'' topics.
Now we return to Chapter 4.3 of the textbook.
Previously we discussed Chapters 4.1-4.2 of the textbook on how regularization relates to overfitting.

\section*{Section 4.3: Validation (+Review)}

The motivating equation of Chapter 4 is
\begin{equation}
    \Eout(h) = \Ein(h) + \text{overfit penalty}.
\end{equation}

%\subsection*{Review: Regularization}

\vspace{2in}

\begin{defn}
Optimization with a \emph{soft order constraint} is defined to be
\begin{equation}
    \label{eq:soc}
    g = \argmin_{h\in\mathcal H} \Ein(h) \qquad \text{subject to}\qquad \Omega(h) \le C
\end{equation}
where $\Omega : \mathcal H \to \R$ is a \emph{regularization function} that penalizes ``complex'' hypotheses,
    and $C : \R$ is a hyperparameter that determines how complex a function is allowed to be.
\end{defn}

\begin{defn}
Define the \emph{augmented error} to be
\begin{equation}
    \Eaug(h) = \Ein(h) + \lambda\Omega(h)
    .
\end{equation}
    Then the \emph{augmented error minimization} problem is
\begin{equation}
    \label{eq:rlm}
    g = \argmin_{h\in\mathcal H} \Eaug(h)
    .
\end{equation}
\end{defn}

\begin{theorem}
    If $\lambda = \Theta(\tfrac 1C)$, then under reasonable conditions, optimizing the augmented error in Eq \eqref{eq:rlm} is equivalent to optimizing the soft order constraint in Eq \eqref{eq:soc}.
\end{theorem}

\begin{fact}
    If $\Omega(h) \approx |\Ein(h) - \Eout(h)|$,
    then $\Eaug \approx \Eout$,
    and $g\approx f$.
\end{fact}

\newpage
\section*{Section 4.3.1: The Validation Set}

%Recall that the dataset is denoted $\mathcal D$ and has size $N$.
%\begin{align}
    %\Dtrain \\
    %\Dval
%\end{align}

\noindent
Illustrate the validation set notation below.

%Equation 4.9 of the textbook states
%\begin{equation*}
    %\Eout(g^-) \le \Eval(g^-) + O(\tfrac1{\sqrt K}).
%\end{equation*}

\vspace{4in}
\noindent
Equation 4.10 of the textbook states
\begin{equation*}
    \Eout(g) \le \Eout(g^-) \le \Eval(g^-) + O\bigg(\sqrt{\frac 1 {K}}\bigg)
    .
\end{equation*}

\vspace{3in}
Describe the difference between a test set and validation set.

\newpage
\section*{Section 4.3.2: Model Selection}

Illustrate the model selection notation below.

\vspace{4in}

\noindent
Equation 4.12 from the textbook states 
\begin{equation*}
    \Eout(g_{m^*}) \le
    \Eout(g^{-}_{m^*}) \le \Eval(g^-_{m^*}) + O\bigg(\sqrt{\frac{\log M}{K}}\bigg).
\end{equation*}

\newpage
\noindent
In 2015, Baidu research (led by Andrew Ng) was accused of cheating on the ImageNet LSVRC challenge.
The lead scientist on the project Dr.\ Wen Ru was fired from his position as ``distinguished scientist at Baidu's Institute of Deep Learning.''

Reference:
\url{https://dswalter.github.io/machine-learnings-first-cheating-scandal.html}

\newpage
\section*{Section 4.3.3: Cross Validation}

The textbook introduces the following un-numbered equation describing the role of $K$:
\begin{equation*}
    \Eout(g) \approx \Eout(g^-) \approx \Eval(g^-).
\end{equation*}

\vspace{3in}
\noindent
Describe cross validation below.

\newpage
\noindent
Consider the following two popular datasets:
\begin{enumerate}
    \item
        The MNIST dataset has 60k data points.
        10k of those data points are reserved for testing
        and 50k data points for training.
        It is traditional to randomly split the training set into a train/validation split with 45k and 5k data points.

    \item
        The ImageNet dataset has 1.2 million data points.
        It is traditional to use 1 million of those data points for training,
        50k for validation,
        and the remaining 150k for testing.
\end{enumerate}
Cross validation is not traditionally used on these data sets.
Why?
\newpage
\section*{Problems}
\begin{problem}
    You are training a logistic regression model to determine whether an image contains a bee or an ant.
    You have about 400 labeled images that have been randomly split into a training set of size 250 and a validation set of size 150.

    \begin{enumerate}
        \item 
            You decide to train two models, one where the inputs to your model are the features generated by the ResNet18 feature map and another where the inputs to your model are the ResNet50 feature map.
            Both feature maps generate the same number of features,
            but the ResNet50 feature map internally has 50 hidden layers.

            For both models, you train for M epochs with a step size of $10^{-3}$ and weight decay of $10^{-4}$.
            You observe that $\Etrain$ for the ResNet18 feature map is 0.18 and $\Etrain$ for the ResNet50 feature map is 0.11.
            Based on this observation, you conclude that the model based on ResNet50 will likely have lower $\Eout$ and you decide to use this as your final model.
            You estimate $\Eout$ by calculating $\Eval$ for the ResNet50 model.
            What can you say about the relationship between $\Eout$ and $\Eval$?

            %What can you say about the relationship between ResNet50's $\Etrain$ and $\Eout$ values?

            \vspace{3in}
        \item
            You decide to only use the ResNet18 feature map to train your model.
            You train the model for M epochs using stochastic gradient descent with a step size of $10^{-3}$ and weight decay of $10^{-4}$.
            After each epoch, you evaluate the model on the validation set.
            Finally, you select the model that had the best validation error.
            What can you say about the relationship between $\Eout$ and $\Eval$?
    \end{enumerate}
\end{problem}

\ignore{
\newpage
\begin{problem}

    \begin{enumerate}
        \item 
            \begin{equation}
                \Etrain(g^-) \le \Ein(g^-)
            \end{equation}
        \item 
            \begin{equation}
                \Etrain(g) \le \Ein(g)
            \end{equation}
        \item 
            \begin{equation}
                \Etrain(g) \le \Etrain(g^-)
            \end{equation}
        \item 
            \begin{equation}
                \Ein(g) \le \Eval(g)
            \end{equation}
        \item 
            \begin{equation}
                \Eout(g) \le \Eval(g^-) + O(\sqrt{\log M/K})
            \end{equation}
    \end{enumerate}
\end{problem}
}

\end{document}



